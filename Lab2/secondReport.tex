%_____________________________________________________________________________
%
%STRUTTURA RELAZIONE:
%
%istituto
%scopo
%introduzione teorica
%strumentazione
%procedimento     
%dati sperimentali con tab
%elaborazione dati
%conclusione
%
%_____________________________________________________________________________

%NOTA UTILE: se in un testo bisogna inserire qualche dato numerico in riga, senza
%dover ricorrere a \begin{equation}, basta includere cio' che serve dentro a $....$
%esempio: $10k\Omega$ per inserire il dato in Ohm, poichè alcuni caratteri non vengono presi se stanno fuori da un'equazione


\documentclass{article}
\usepackage{amsmath}
\usepackage{setspace}
\usepackage{anysize}
\usepackage{geometry}
\usepackage{epsfig}
\usepackage{graphicx}
%c'è 1 inch di margine a destra e sinistra
%\geometry{margin = 1.25 in}

\title{\huge Relazione seconda esperienza di laboratorio Fisica 2}
\author{Gruppo A15: Armani Stefano, Cappellaro Nicola, Pasquato Leonardo}
\date{07-11-2022}
\setlength{\parindent}{0cm}

\begin{document}
    %print sezione titolo
    \maketitle
    \rule{\linewidth}{0.1mm}

    \section{Scopo dell'esperienza}
    
    
    \section{Cenni teorici}

    \section{Strumentazione}
    \begin{itemize}
        \item Breadboard con annessi morsetti serrafilo;
        \item Cavi con connettori a banana;
        \item Resistori di varie misure ($1k\Omega$, $10k\Omega$, $100k\Omega$);
        \item Capacitori da $10nF$, $100nF$;
        \item Connettori da banco (Jumper);
        \item Generatore di forme d'onda;
        \item Oscilloscopio.
    \end{itemize}

    \section{Esperimento}
    
    \section{Dati sperimentali}

    \section{Elaborazione dati}
    
    \section{Conclusione}
    
\end{document}