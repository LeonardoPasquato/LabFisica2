%____________________________________________________________________________
%
%STRUTTURA RELAZIONE:
%
%istituto
%scopo
%introduzione teorica
%strumentazione
%procedimento     
%dati sperimentali con tab
%elaborazione dati
%conclusione
%
%_____________________________________________________________________________

%NOTA UTILE: se in un testo bisogna inserire qualche dato numerico in riga, senza
%dover ricorrere a \begin{equation}, basta includere cio' che serve dentro a $....$
%esempio: $10k\Omega$ per inserire il dato in Ohm, poichè alcuni caratteri non vengono presi se stanno fuori da un'equazione


\documentclass{article}
\usepackage{amsmath}
\usepackage{setspace}
\usepackage{anysize}
\usepackage{geometry}
\usepackage{epsfig}
\usepackage{graphicx}
\usepackage{xcolor}
\usepackage{caption}
\usepackage{geometry}
\geometry{a4paper, top=3cm, bottom=3cm, left=2.5cm, right=2.5cm, bindingoffset=5mm}
%\captionsetup[table]{position=top, labelformat=empty}
%c'è 1 inch di margine a destra e sinistra
%\geometry{margin = 1.25 in}

\title{ Relazione terza esperienza di laboratorio Fisica 2}
\author{Gruppo A15: Armani Stefano, Cappellaro Nicola, Pasquato Leonardo}
\date{07-11-2022}
\setlength{\parindent}{0cm}

\begin{document}
    %print sezione titolo
    \maketitle
    \rule{\linewidth}{0.1mm}

    \section{Scopo dell'esperienza}

    \section{Cenni teorici}

    \section{Strumentazione}
    \begin{itemize}
        \item Breadboard con annessi morsetti serrafilo;
        \item Cavi con connettori a banana e connettori da banco (Jumper);
        \item Resistori di varie misure ($1k\Omega$, $10k\Omega$), capacitori da $1nF$ $10nF$, $100nF$;
        \item decade di induttanze
        \item Generatore di forme d'onda Rigol DG1032;
        \item Oscilloscopio Rigol MSO2102A.
    \end{itemize}

    \section{Esperimento}
    Nella terza esperienza di laboratorio è stato costruito manualmente un semplice circuito LTI di secondo ordine, ossia
    un circuito composto da generatore di forme d'onda, capacitore, induttore e resistore. Per poter inserire un induttore, è stato utilizzato un dispositivo
    detto $decade di induttanze$, il quale mette a disposizione 10 induttori variabili in serie. \par
    Il circuito in questione è il seguente : %figura
    \par
    Nel primo esperimento è stato dato in ingresso a questa rete un segnale sinusoidale con offset nullo e 
    tensione picco-picco $V^{pp}_{in} = 5V$, di cui però è stata variata la frequenza più volte per ottenere diverse misurazioni
    di ampiezza e sfasamento della tensione d'uscita sul resistore, quindi $\mathbf{V}_R$.
    Una volta ottenute le misurazioni è possibile approssimare la funzione di trasferimento sperimentale $H_sp(j\omega)$
    e confrontarla con la funzione di trasferimento teorica $H(j\omega)$.\par
    È stata ripetuta questa procedura dopo aver sostituito il resistore corrente con uno avente una resistenza
    pari a $1k\Omega$. \par
    Durante il secondo esperimento è stato utilizzato lo stesso circuito, di cui sono state utilizzate 3 terne di valori di
    resistenze, induttanze e capacità. Dopo aver fornito in ingresso un'onda quadra di tensione picco-picco $V^{pp}_{in} = 2.5 V$
    e offset $V^{of}_{in} = 1.250 V$.
    
    \section{Dati sperimentali}

    \section{Elaborazione dati}

    \section{Conclusione}

\end{document}